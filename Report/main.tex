\documentclass{beamer}

\mode<presentation> {
    \usepackage[utf8]{inputenc}
    \usepackage[T1]{fontenc}
    \usepackage[english]{babel}
    \usepackage{csquotes}
    \usepackage{ragged2e} 

    \bibliographystyle{plain}

    \usetheme{Madrid}
    \usecolortheme{ipbutfpr}

    % Can change the colors at beamercolorthemeipbutfpr.sty
    %\setbeamersize{text margin left=0.6cm,text margin right=0.6cm}
    \setbeamertemplate{caption}[numbered]
    \setbeamertemplate{bibliography item}{\insertbiblabel}
    \setbeamertemplate{frametitle continuation}{\gdef\beamer@frametitle{}\justifying}

    \setbeamertemplate{footline}{
        \leavevmode%
        \hbox{%
            \begin{beamercolorbox}[wd=.5\paperwidth,ht=2.25ex,dp=1ex,center]{author in head/foot}%
                \usebeamerfont{author in head/foot}\insertshortauthor
            \end{beamercolorbox}%
            \begin{beamercolorbox}[wd=.5\paperwidth,ht=2.25ex,dp=1ex,center]{title in head/foot}%
                \hspace*{7em}\usebeamerfont{title in head/foot}\insertshorttitle\hspace*{7em}
                \insertframenumber{} / \inserttotalframenumber\hspace*{1ex}
            \end{beamercolorbox}
        }%
        \vskip0pt%
    }

    \setbeamertemplate{navigation symbols}{}
}

\usepackage{amsmath}
\usepackage{amssymb}
\usepackage[language=english, style=numeric, sorting=none]{biblatex}

\addbibresource{main.bib}
\makeatletter
\makeatother

\title{\textbf{Graver Basis}}
%\subtitle{A short story}

\author{\textbf{Francisco Javier Blázquez Martínez}}

\institute[EPFL]{\normalsize
    % TODO: How should this be?
    % Prof. Friedrich Eisenbrand \\
    % Jana Cslovjecsek \\

    \begin{figure}[htb]
        % TODO: Maybe shift it a little bit to the right
        \centering
        \includegraphics[width=0.4\textwidth]{images/logos/epfl_logo.png}
    \end{figure}
}

% TODO: Chair of discrete optimization is maybe too much
\date{\small Chair of discrete optimization \\ November 2020}

\begin{document}
    \justifying
    
    \frame{\titlepage}
    
    % TODO: Index or not necessary?
    %\begin{frame}
    %\frametitle{Table of Contents}
    %\tableofcontents
    %\end{frame}

    \section{Integer Linear Programming}
    \begin{frame}
        \frametitle{Integer Linear Programming}
        The underlying question is how to solve the integer linear problem (IP).
        \begin{equation*}
            (IP) \equiv max\{c^tx : Ax = b, l \leq x \leq u, x \in \mathbb{Z}^n \}
        \end{equation*}
        %\pause
        Where $A \in \mathbb{Z}^{mxn}$, $b \in \mathbb{Z}^m$, $c \in \mathbb{Z}^n$, $l$ and $u$ lower and upper bounds for x.
        %\pause
        
        \vspace{1cm}
        \alert{IP is NP-Hard}. There are algorithms with polynomial complexity for certain IPs and there are algorithms for the general IP based on cutting plane methods, dinamic programming, lattice-basis reduction... Thanks to the study of Graver basis new algorithms have appeared improving classic techniques in certain cases. 
        
        
        
    \end{frame}
    
    \begin{frame}
        \frametitle{Integer Linear Programming}

    \end{frame}
    
    \section{Graver Basis, definition and properties}
    \begin{frame}
        \frametitle{Graver Basis}
        \begin{itemize}
            \item \textbf{Definition 1:} Two vectors $u,v \in \mathbb{R}^n$ are said to be \textbf{sign compatible} if $u_i \cdot v_i \geq 0$ for all $i \in \{1,...,n\}$
            \vspace{0.2cm}
            %\pause
            \item \textbf{Definition 2:} A vector $u \in ker(A)$ is \textbf{indecomposable} if it is not the sum of two sign compatible and non zero elements in $ker(A)$.
            %\pause
        \end{itemize}
        \vspace{1cm}
        \begin{block}{Graver Basis }
            The Graver Basis of a given matrix A is defined as the set of integral indecomposable elements in the kernel of A.
        \end{block}
    \end{frame}
    
    \begin{frame}
        \frametitle{Graver Basis properties}
        \begin{itemize}
            \item \textbf{Property 1:} Every integral element in ker(A) can be expressed as positive integral linear combination of elements in GR(A).
            \vspace{0.2cm}
            %\pause
            \item \textbf{Property 2:} Given z in the feasible region of an IP, z is not optimum if and only if there exists $g \in G(A)$ s.t. $c^tg > 0$ and $l \leq z + g \leq u$
            %\pause
        \end{itemize}
        
    \end{frame}
    
    \begin{frame}
        \frametitle{Graver Basis bounds}
        The \textbf{Property 2} we have seen is an optimality test which has the advantage that only needs to consider the Graver basis elements. An important 
    \end{frame}
    
    \begin{frame}
        \frametitle{Algorithm}
        \begin{enumerate}
            \item Given a feasible solution $z_0$
            \item Solve the subproblem:\\
                  $max\{c^tg : Ag = 0, l-z_0 \leq g \leq u-z_0, g \in \mathbb{Z}^n, ||g||_1 \leq ||G(A)|| \}$
            \begin{itemize}
                \item $g^* = 0 \implies z_0$ optimal solution.
                \item $g^* \neq 0 \implies$ g* improvement direction, loop back to 1 with $z_1 = z_0 + \lambda \cdot g^*$ with the biggest $\lambda$ respecting the bounds.
            \end{itemize}
        \end{enumerate}
    \end{frame}
    
    \section{N-Fold, a success example}
    \begin{frame}
        \frametitle{One example of success, N-Fold}
        
        A N-Fold IP has constriction matrix A of the form ($A_i \in \mathbb{Z}^{rxt}, B_i \in \mathbb{Z}^{sxt}$):\\
        \begin{equation*}
        N = 
        \begin{pmatrix}
        A_1 & A_2 & \cdots & A_n \\
        B_1 & 0   & \cdots & 0 \\
        0   & B_2 & \cdots & 0 \\
        \vdots    & \vdots & \ddots & \vdots  \\
        0   & 0   & \cdots & B_n 
        \end{pmatrix}
        \end{equation*}
        
        \begin{block}{N-Fold Graver basis bound}
            For all $g \in G(A)$ $||g||_1 \leq L_B (2r\Delta L_B + 1)^r =: L_A$ 
        \end{block}
        

    \end{frame}
    
    
    
    %----SLIDES----%
    %\include{capitulos/introducao}
    %\include{capitulos/referencial-teorico}
    %\include{capitulos/metodologia}
    %\include{capitulos/resultados}
    %\include{capitulos/conclusoes}
    %--------------%

    \section{References}
    \begin{frame}[allowframebreaks] % Interesting option!
        \frametitle{References}
        \nocite{*}
        \printbibliography
    \end{frame}

\end{document}
