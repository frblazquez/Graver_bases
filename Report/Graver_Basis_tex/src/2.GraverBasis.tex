\chapter{Graver Basis} \label{literature}

%\definition  % This also works
\begin{definition}
Two vectors $u,v \in \mathbb{R}^n$ are said to be \textbf{sign compatible} if $u_i \cdot v_i \geq 0$ for all $i \in \{1,...,n\}$.
\end{definition}

\begin{definition}
A vector $u \in ker(A)$ is \textbf{indecomposable} if it is not the sum of two sign compatible and non zero elements in $ker(A)$.
\end{definition}

\begin{definition}[\textbf{Graver basis}]
The Graver Basis of a given matrix $A \in \mathbb{Z}^{mxn}$ is defined as the set of integral indecomposable elements in the kernel of A.\\
(Initially defined as \textit{universal integral test set} in [Graver 1975])
\end{definition}


% \textbf{Graver basis definition}
% \begin{itemize}
%     \item \textbf{Definition:}
%     Two vectors $u,v \in \mathbb{R}^n$ are said to be \textbf{sign compatible} if $u_i \cdot v_i \geq 0$ for all $i \in \{1,...,n\}$.
%     \vspace{0.2cm}
%     \item \textbf{Definition:} A vector $u \in ker(A)$ is \textbf{indecomposable} if it is not the sum of two sign compatible and non zero elements in $ker(A)$.
%     \item \textbf{Graver Basis $\equiv Gr(A)$: }
%     The Graver Basis of a given matrix A is defined as the set of integral indecomposable elements in the kernel of A.\\
%     (Initially defined as \textit{universal integral test set} in [Graver 1975])
% \end{itemize}

\begin{proposition}
For every matrix A, $Gr(A)$ is a finite set.
\end{proposition}

\begin{proposition}
Every integral element in ker(A) can be expressed as positive integral linear combination of elements in $Gr(A)$.
\end{proposition}

\begin{proposition}
Given z in the feasible region of an IP, z is not optimum if and only if there exists $g \in Gr(A)$ s.t. $c^tg > 0$ and $l \leq z + g \leq u$
\end{proposition}

\begin{proposition}[\textbf{Graver basis bounds}]
Given $A \in \mathbb{Z}^{mxn}$ and $\Delta$ an upper bound for the absolute value of each component of $A$, for every $g \in Gr(A)$:
\begin{itemize}
    \item $||g||_1 \leq m^{m/2}\Delta^m\cdot(n - m)$ \hspace{10pt}[Onn 2010]
    \item $||g||_1 \leq (2m \Delta + 1)^m$ \hspace{41pt}[Eisenbrand,Hunkenschröder,Klein 2018]
\end{itemize}
\end{proposition}

\textbf{Bases of augmentation algorithm}
\begin{itemize}
    \item If not optimal, an element in Graver basis is an improvement direction.
    \item If Graver basis bounded, we can restrict our improvement direction search.
\end{itemize}

\textbf{General IP algorithm using Graver basis norm bound}
\begin{enumerate}
    \item From a feasible solution $z_i$
    \item Find $g^*$ optimum for the sub-problem: \vspace{4pt}\\
          $max\{c^tg : Ag = 0, l-z_i \leq g \leq u-z_i, g \in \mathbb{Z}^n, ||g||_1 \leq ||Gr(A)|| \}$ \vspace{4pt}
    \begin{itemize}
        \item $g^* = 0 \implies z_i$ optimal solution.
        \item $g^* \neq 0 \implies$ $g^*$ improvement direction, loop back to 1 with $z_{i+1} = z_i + \lambda \cdot g^*$ with the biggest $\lambda$ respecting the bounds.
    \end{itemize}
\end{enumerate}
\hspace{15pt} [Hemmecke, Onn, Romanchuk 2013]


        
