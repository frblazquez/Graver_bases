\chapter{N-Fold IP} \label{methodology}

\lhead{\emph{N-Fold IP}}  % Set the left side page header to "Contents"

% De Loera, Hemmecke, Onn and Weismantel 2006 paper


A generalized N-Fold IP has constriction matrix A of the form ($A_i \in \mathbb{Z}^{rxt}, B_i \in \mathbb{Z}^{sxt}$):\\
\begin{equation*}
N = 
\begin{pmatrix}
A_1 & A_2 & \cdots & A_n \\
B_1 & 0   & \cdots & 0 \\
0   & B_2 & \cdots & 0 \\
\vdots    & \vdots & \ddots & \vdots  \\
0   & 0   & \cdots & B_n 
\end{pmatrix}
\end{equation*}

% TODO - ADD:
% Presentation of the N-Fold IP (generalized or not??)
% Particularities
% Steinitz Lemma? Or beter in next section?
% Theoretical justification of N-Fold being polynomially solvable

% TODO - DOUBT:
% I think this propositions are not working for the generalized N-Fold!!

% TODO: 
% General algorithm, maybe it should go to the previous section
% Explain that the trick is that the Graver basis of A is a parameter


% Moved to greedy augmentation algorithm in section 2!
%\begin{proposition}
%There is a polynomial time algorithm that, given any matrix $A \in \mathbb{Z}^{m \times n}$ along with its Graver Basis $G(A)$, and vectors $x \in \mathbb{N}^n$ and $c \in \mathbb{Z}^n$ solves the integer program $IP_A(b,c)$. 
%\end{proposition}

The N-Fold IP has a withe range of applications in ... (See that transportation problems etc admit N-Fold formulation!). It's also very interesting by itself by the theoretical properties it has.

\section{N-Fold properties}

\begin{proposition}
Fix any pair of integer matrices $A \in \mathbb{Z}^{r \times q}$ and $B \in \mathbb{Z}^{s \times q}$. Then there is a polynomial time algorithm that, given n, computes the Graver basis of the N-Fold matrix $[A,B]^{(n)}$. In particular, the cardinality and the bit size of $G([A, B]^(n))$ are bounded by a polynomial function of n.
\end{proposition}

\begin{proposition}
Fix any pair of integer matrices $A \in \mathbb{Z}^{r \times q}$ and $B \in \mathbb{Z}^{s \times q}$. Then there is a polynomial time algorithm that, given n, objective vector $c \in \mathbb{Z}^{nq}$, and non-negative integer vector $x \in \mathbb{Z}^{nq}$ , solves the generalized N-Fold integer programming problem in which x is feasible.
\end{proposition}

\begin{proposition}
Fix any pair of integer matrices $A \in \mathbb{Z}^{r \times q}$ and $B \in \mathbb{Z}^{s \times q}$. Then there is a polynomial time algorithm that, given n and demand vector $b \in \mathbb{Z}^{s + nr}$, either finds a feasible point $x \in \mathbb{N}^{nq}$ to the N-Fold IP of order n, or asserts that no feasible solution exists.
\end{proposition}


\begin{theorem}[\textbf{N-Fold IP is polynomially solvable}]
Fix any pair of integer matrices A, B of compatible sizes. Then there is a polynomial time algorithm that solves the generalized n-fold integer programming problem on any input n, b, c.
\end{theorem}

% TODO: No clear this part at all!