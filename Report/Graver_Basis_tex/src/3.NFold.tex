\chapter{N-Fold IP} \label{3.N-Fold}
\lhead{\emph{N-Fold IP}}  

% N-Fold introduction and notation
A generalized \emph{N-Fold IP} has constraint matrix A of the form ($A_i \in \mathbb{Z}^{rxt}, B_i \in \mathbb{Z}^{sxt}$):\\
\begin{equation*} \label{NFold_matrix}
\mathcal{N} = 
\begin{pmatrix}
A_1 & A_2 & \cdots & A_n \\
B_1 & 0   & \cdots & 0 \\
0   & B_2 & \cdots & 0 \\
\vdots    & \vdots & \ddots & \vdots  \\
0   & 0   & \cdots & B_n 
\end{pmatrix}
\end{equation*}


It was presented in \cite{LHOW:2006} in 2006 in a simplified version in which $\forall i,j$  $A_i = A_j, B_i = B_j$. This simplified N-Fold matrix is totally determined given $A \in \mathbb{Z}^{r \times t}$, $B \in \mathbb{Z}^{s \times t}$ and the order $n$ and we denote it by $N_{A,B}^{(n)}$. Hereafter, we'll refer to the generalized formulation as \emph{N-Fold IP}. 

% N-Fold relevance
The \emph{N-Fold IP} has a wide range of applications. In \cite{LHOW:2006} it's applied to the multiway transportation and cutting-stock problems and in \cite{HEMMECKE:2011} to privacy and disclosure control in
statistical databases just to name a few examples beyond the typical in operations research. In fact, the \emph{N-Fold IP} is universal. Every IP can be expressed as an \emph{N-Fold IP}. This is because, as shown in \cite{LO:2006}, every IP can be modeled as a slim 3-way transportation program, which can be expressed as an \emph{N-Fold IP}. However, this is more a theoretical achievement which shows the expressiveness power of the N-Fold rather than a useful practical approach for solving any IP. 

In any case, the \emph{N-Fold IP} is interesting by itself since it has good theoretical properties. In the following sections we will study it with the help of Graver bases to finally obtain a roughly linear algorithm for its resolution. We start with the following proposition:

% PROPERTIES
% N-Fold graver basis is polynomial
\begin{theorem} \label{Nfold_GB_computation_polynomial}
Fixed any $A \in \mathbb{Z}^{r \times t}$ and $B \in \mathbb{Z}^{s \times t}$, there is an algorithm polynomial in $n$ that computes the Graver basis of the N-Fold matrix $N_{A,B}^{(n)}$. 
\end{theorem}
\vspace{-10pt}

We won't go into the details of the proof of this proposition, they can be seen in \cite[Theorem 4.2]{LHOW:2006}. However, we consider important to remark that this proposition implies that, fixed $A$ and $B$, the cardinality of $\mathcal{G}(N_{A,B}^{(n)})$ is bounded by a polynomial function of $n$.

This has an important consequence, the greedy Graver basis augmentation algorithm presented before has, in this case, polynomial complexity for every augmentation step and therefore polynomial complexity. However it's still remaining the problem of obtaining an initial feasible solution. That is precisely what solves the next proposition:

% Two phase method for N-Fold
\begin{lemma} \label{Nfold_two_phase}
Fixed any $A \in \mathbb{Z}^{r \times t}$ and $B \in \mathbb{Z}^{s \times t}$, there is an algorithm polynomial in $n$ that, given a  demand vector $b \in \mathbb{Z}^{s + nr}$, either finds a feasible point $x \in \mathbb{Z}^{nt}$ to the simplified \emph{N-Fold IP} of order n, or asserts that no feasible solution exists.
\end{lemma}
\vspace{-15pt} 
\begin{proof}
This is a result proved in \cite[Lemma 5.2]{LHOW:2006}. The idea is that we can add $2n(s+r)$ artificial variables to construct an \emph{N-Fold IP} for which an initial feasible solution is trivial for any right side $b$. The constraint matrix would result as below: 
\vspace{5pt}
\begin{equation*}
N = 
\setcounter{MaxMatrixCols}{20}
\begin{pmatrix}
A & I_s & -I_s & 0 & 0 & A & I_s & -I_s & 0 & 0 & \cdots & A & I_s & -I_s & 0 & 0 \\
B & 0 & 0 & -I_r & I_r & 0 & 0 & 0 & 0 & 0 & \cdots & 0 & 0 & 0 & 0 & 0 \\
0 & 0 & 0 & 0 & 0 & B & 0 & 0 & -I_r & I_r & \cdots & 0 & 0 & 0 & 0 & 0 \\
\vdots & \vdots & \vdots & \vdots & \vdots & \vdots & \vdots & \vdots & \vdots & \vdots & \ddots & \vdots & \vdots  & \vdots & \vdots & \vdots \\
0 & 0 & 0 & 0 & 0 & 0 & 0 & 0 & 0 & 0 & \cdots & B & 0 & 0 & -I_r & I_r 
\end{pmatrix}
\end{equation*}

This way the result is obtained applying the \textit{two phase method} which, thanks to the previous proposition, runs in polynomial time. Note that we need to add $I$ and $-I$ terms because the artificial variables have to be positive.
\vspace{5pt}
\end{proof}

\begin{theorem}[\textbf{Simplified N-Fold IP is polynomially solvable}]
Fix any pair of integer matrices $A \in \mathbb{Z}^{r \times t}$ and $B \in \mathbb{Z}^{s \times t}$. Then there is a polynomial time algorithm that solves the simplified N-Fold integer programming problem on any input n, b, c.
\end{theorem}
\vspace{-15pt} % TODO: Should be 20
\begin{proof}
Propositions \ref{Nfold_GB_computation_polynomial} and \ref{Nfold_two_phase} ensure that we can compute the  Graver basis of $N_{A,B}^{(n)}$ and a initial feasible point in polynomial time for any $n$, $b$ and $c$. Applying the algorithm \ref{GB_greedy_algorithm} after this solves the problem in global polynomial time.
\end{proof}

% TODO: Maybe too much blank space here!
