\chapter{Introduction} \label{1.Introduction}
\lhead{\emph{Introduction}}  

% Present the IP problem
Hereafter, the underlying problem is the classical \textit{Integer Program} (IP), that we formulate in the following way:
\begin{equation*}
    (IP) \equiv max\{c^tx : Ax = b, l \leq x \leq u, x \in \mathbb{Z}^n \}
\end{equation*}
\vspace{-50pt}
\begin{center}
$A \in \mathbb{Z}^{m \times n}$, $b \in \mathbb{Z}^m$, $c \in \mathbb{Z}^n$, $l$  and $u$ lower and upper bounds for $x$
\end{center}


% General introduction
Despite the simplicity of its formulation, allowing only linear constraints and a linear objective function, the importance of IP is well known . A large number of problems in diverse fields of the mathematics and algorithms with an infinity of applications admit an IP formulation. Unfortunately, it's also well known that integer programming is NP-Complete, what means that no efficient algorithm is likely to exist for solving an IP in the general case. This explains the great interest in restricted formulations of the problem and in certain resolution techniques even when they can't be applied to the general IP. In the following sections we present the latest techniques based on the \textbf{Graver bases} and its bounds as well as their application to the \textbf{N-Fold IP}, a restricted formulation of the IP which has won relevance in the last decades given its theoretical properties and its wide applications. 

% Document structure
For this purpose, in chapter \ref{2.Graver_bases} we introduce the Graver basis of a given matrix, explore its properties, bounds, and how these can be applied for solving the general IP. We then study in chapter \ref{3.N-Fold} the N-Fold case and show, with the help of Graver bases, that the N-Fold IP can be solved in polynomial time, delving into the best known algorithm for this case.


% We could add much more here if needed:
%
% We can briefly introduce each section/chapter
% Explicit references to Graver Basis appearance
% Explicit references to N-Fold appearance
% Explicit general techniques for IP
% Explicit specific techniques for restricted IP

