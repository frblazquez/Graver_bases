\chapter{Introduction} \label{introduction}

% Present the IP problem
The underlying problem we try to solve is the clasical \textit{Integer program} (IP) that we formulate in the following way:
\begin{equation*}
    (IP) \equiv max\{c^tx : Ax = b, l \leq x \leq u, x \in \mathbb{Z}^n \}
\end{equation*}
\vspace{-50pt}
\begin{center}
$A \in \mathbb{Z}^{mxn}$, $b \in \mathbb{Z}^m$, $c \in \mathbb{Z}^n$, $l$  and $u$ lower and upper bounds for x
\end{center}

% TODO - ADD:
% Simplicity: Not allowing non-linear restrictions/objective functions
% Powerful: Lot of applications, many problems admit IP formulation
% Problem: Is NP-Complete
% Solution (+-): Techniques for certain IPs, GRAVER BASIS!
% N-Fold: As a success example

Despite its simplicity, it's well known the importance of IP. Several problems in diverse fields of the mathematics and algorithms admit an IP equivalent formulation (examples?). Unfortunately, IP is NP-Complete. This means that there is no efficient (polynomial) algorithm for solving IP in the general case (say general techniques and complexity?) and, therefore, knowing their importance and the lack of a general efficient algorithm for their resolution, there has been a great interest in restricted formulations of the problem and their resolution techniques.

In this project we present the concept of \textbf{Graver Basis} and its applications for solving the IP with, of course, the theoretical justification of this based on its properties. We apply this to a concrete IP formulation, the N-Fold IP and prove that it leads to a polynomial and efficient algorithm for this case.













