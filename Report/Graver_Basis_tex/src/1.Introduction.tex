\chapter{Introduction} \label{introduction}

% TODO: Set headers!!
\lhead{\emph{Introduction}}  % Set the left side page header to "Contents"

% Present the IP problem
The underlying problem is the clasical \textit{Integer program} (IP) that we formulate in the following way:
\begin{equation*}
    (IP) \equiv max\{c^tx : Ax = b, l \leq x \leq u, x \in \mathbb{Z}^n \}
\end{equation*}
\vspace{-50pt}
\begin{center}
$A \in \mathbb{Z}^{mxn}$, $b \in \mathbb{Z}^m$, $c \in \mathbb{Z}^n$, $l$  and $u$ lower and upper bounds for x
\end{center}

% TODO - ADD:
% Simplicity: Not allowing non-linear restrictions/objective functions
% Powerful: Lot of applications, many problems admit IP formulation
% Problem: Is NP-Complete
% Solution (+-): Techniques for certain IPs, GRAVER BASIS!
% N-Fold: As a success example

Despite the simplicity of its formulation, allowing only linear constraints and a linear objective function, it's well known the importance of IP. A large number of problems in diverse fields of the mathematics and algorithms (with an infinity of applications) admit an IP formulation. Unfortunately, it's also well known that IP is NP-Complete, what means that no efficient algorithm is likely to exist for solving IP in the general case. This explains the great interest in restricted formulations of the problem and in certain resolution techniques (even when they are not useful for the general IP). % One of this techniques is decomposing the constraint matrix into block structure which has proved useful for IP formulations where this matrix is sparse or has a certain shape.

In this project we first present the concept of \textbf{Graver Basis}, its properties, and its applications for solving the IP. We then study the N-Fold IP, a restricted formulation which has won relevance in the last decades given its theoretical properties and its wide applications. We show with the help of Graver Basis that N-fold can be solved in polynomial time. In chapters four and five we go further for obtaining a better complexity and show how applying these Graver Basis techniques (and specially its bounds) we can restrict the search of an improvement direction at a feasible point or even restrict the search to an optimal solution at a solution of the linear relaxation. All together lead us to a polynomial and efficient algorithm for this kind of problems.


% We could add much more here if needed:
%
% We can briefly introduce each section/chapter
% Explicit references to Graver Basis appearance
% Explicit references to N-Fold appearance
% Explicit general techniques for IP
% Explicit specific techniques for restricted IP








